% "Лабораторная работа"

\documentclass[a4paper,12pt]{article} % тип документа

% report, book

%  Русский язык

\usepackage[T2A]{fontenc}			% кодировка
\usepackage[utf8]{inputenc}			% кодировка исходного текста
\usepackage[english,russian]{babel}	% локализация и переносы


% Математика
\usepackage{amsmath,amsfonts,amssymb,amsthm,mathtools} 


\usepackage{wasysym}

%Заговолок
\author{Иванов Дмитрий, ИВТ3}
\title{Основы работы в \LaTeX{}}
\date{\today}


\begin{document} % начало документа

\maketitle
\newpage

\section{Издательские системы}
Для подготовки буклетов, оформления журналов и книг предназначены специальные издательские системы. Они позволяют готовить их и печатать на принтерах или выводить на фотонаборные автоматы сложные документы высокого качества.

\textit{Настольные издательские системы (НИС)} — это программы, предназначенные для профессиональной издательской деятельности, позволяющие осуществлять электронную верстку широкого спектра основных типов документов.

Известными пакетами среди издательских систем для компьютеров являются PageMaker, QuarkXPress, Scribus и др.

\subsection{Издательская система TeX}
\TeX — система компьютерной вёрстки, разработанная американским профессором информатики Дональдом Кнутом в целях создания компьютерной типографии. В неё входят средства для секционирования документов, для работы с перекрёстными ссылками. Многие считают TeX лучшим способом для набора сложных математических формул. В частности, благодаря этим возможностям, TeX популярен в академических кругах, особенно среди математиков и физиков.

\subsection{Дональд Кнут}
Дональд Эрвин Кнут (род. 10 января 1938 года, Милуоки, штат Висконсин) — американский учёный в области информатики, эмерит-профессор Стэнфордского университета, профессор СПбГУ и других университетов, преподаватель и идеолог программирования, автор 19 монографий (в том числе ряда классических книг по программированию) и более 160 статей, разработчик нескольких известных программных технологий. Автор всемирно известной серии книг, посвящённой основным алгоритмам и методам вычислительной математики, а также создатель настольных издательских систем \TeX и METAFONT, предназначенных для набора и вёрстки книг научно-технической тематики (в первую очередь — физико-математических).


\subsection{Издательская система LaTeX}
\LaTeX{} — наиболее популярный набор макрорасширений (или макропакет) системы компьютерной вёрстки \TeX, который облегчает набор сложных документов. В типографском наборе системы \TeX форматируется традиционно как \LaTeX{}.
Важно заметить, что ни один из макропакетов для \TeX’а не может расширить возможностей \TeX (всё, что можно сделать в \LaTeX{}’е, можно сделать и в \TeX’е без расширений), но, благодаря различным упрощениям, использование макропакетов зачастую позволяет избежать весьма изощрённого программирования.


\subsection{Лесли Лэмпорт}
Лесли Лэмпорт (англ. Leslie Lamport; 7 февраля 1941 года, Нью-Йорк) — американский учёный в области информатики, первый лауреат премии Дейкстры. Разработчик \LaTeX{} — популярного набора макрорасширений системы компьютерной вёрстки \TeX, исследователь теории распределённых систем, темпоральной логики и вопросов синхронизации процессов во взаимодействующих системах. Лауреат Премии Тьюринга 2013 года.


\section{Основные правила создания текстового документа}
\LaTeX{} использует специальный язык разметки, преобразуя исходный текст вместе с его разметкой в документ высокого качества. Аналогичным образом формируются веб-страницы: исходный текст записывается с помощью языка HTML, а браузер открывает эту страницу уже во всей красе — с различными цветами, шрифтами, размерами и т.д.

Процесс создания документов с системе LaTeX состоит из следующих этапов:
\begin{itemize}
\item В LaTeX-редакторе создать исходный файл (LaTeX-файл) – файл с расширением .tex (например, hi.tex), который содержит текст документа и специальные команды, указывающие LaTeX, как именно нужно сверстать этот текст.
\item Скомпилировать исходный файл (hi.tex) в файл документа, например в формате PDF (hi.pdf) с помощью PDFLaTeX или XeLaTeX.
\item Посмотреть результирующий файл. Если результат устраивает, распечатать его. Иначе внести изменения в исходный файл, снова скомпилировать его и т.д.


\end{itemize}

\end{document} % конец документа