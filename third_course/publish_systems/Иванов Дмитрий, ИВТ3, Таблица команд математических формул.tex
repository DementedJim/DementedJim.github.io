\documentclass[a4paper,12pt]{article} % тип документа
% report, book
%  Русский язык
\usepackage[T2A]{fontenc}			% кодировка
\usepackage[utf8]{inputenc}			% кодировка исходного текста
\usepackage[english,russian]{babel}	% локализация и переносы
% Математика
\usepackage{amsmath,amsfonts,amssymb,amsthm,mathtools} 
\usepackage{wasysym}
\usepackage{array}
\usepackage{hyperref}
%Заговолок
\author{Иванов Дмитрий, ИВТ3}
\title{Таблица математических команд \LaTeX{}}
\date{\today}

\begin{document} % начало документа
\maketitle
\newpage


\flushleft
\begin{tabular}{|l|l|}
  \hline
  \multicolumn{2}{|c|}{Команды} \\
  \hline
  Назначение команды & $\backslash$Вид \\
  \hline
  Формула  & \$ x+y=z \$ \\
  \hline
  Знаки умножения  & cdot и times  \\
  \hline
  Корни  &  sqrt[n]\{подкоренное выражение\} \\
  \hline
   Символ бесконечности  &  infty \\
   \hline
   Неравенства  & \&lt \&gt ge le\\
   \hline
    Дроби & \$ a+b/c \$ \\
   \hline
  Числитель над знаменателем  & \$ a+frac\{b\}\{c\} \$  \\
  \hline
  Степени & \$ a\^2, b\^\{34\}, r\^ \{abs\}, k\^ \{l\^ \{2\}\} \$ \\
  \hline
  Индексы & \$ a\_1, b\_\{34\}, r\_\{abs\}, k\_\{l\_\{1\}\} \$ \\
  \hline
  степени и индекс одновременно & \$ a\_3\^2  \$ \\
  \hline
  формула посредине отдельной строки & \$\$ x+y=z \$\$ \\
  \hline
  Пределы  & lim, стрелки to\\
  \hline
  Многоточие  & ldots, ddots, vdots\\
  \hline
  Матрица  &  begin\{array\}\{clr кол-во столб.\} 1 \& 2 end\{array\}  \\
  \hline
\end{tabular}

\end{document} % конец документа