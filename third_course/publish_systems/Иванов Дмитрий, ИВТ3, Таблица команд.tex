\documentclass[a4paper,12pt]{article} % тип документа

% report, book

%  Русский язык

\usepackage[T2A]{fontenc}			% кодировка
\usepackage[utf8]{inputenc}			% кодировка исходного текста
\usepackage[english,russian]{babel}	% локализация и переносы


% Математика
\usepackage{amsmath,amsfonts,amssymb,amsthm,mathtools} 


\usepackage{wasysym}

\usepackage{hyperref}

%Заговолок
\author{Иванов Дмитрий, ИВТ3}
\title{Таблица команд для форматирования технического текста в \LaTeX{}}
\date{\today}


\begin{document} % начало документа
\maketitle
\newpage

\begin{tabular}{ | l | l | }
\hline
Назначение команды & Вид (написание) команды \\ \hline
\tiny{Самый маленький размер шрифта} & \string\tiny \\
\scriptsize{Меньше меньшего размер шрифта} & \string\scriptsize \\
\footnotesize{Меньший размер шрифта} & \string\footnotesize \\
\small{Маленький шрифта} & \string\small\\
Символ \# \ & \string\# или \string\textnumbersig \\
Символ \$ \ & \string\$ или \string\textdollar \\
Символ \textasciicircum \ & \string\textasciicircum \\
Символ \& \ & \string\& или \string\textampersand \\
Символы \{ и \} \ & \string\{ и \string\} соответсвенно \\
Символ \textasciitilde \ & \string\textasciitilde \\
Игнорирование разрыва строки & \string\textit \\

\hline

\end{tabular}
\end{document}
