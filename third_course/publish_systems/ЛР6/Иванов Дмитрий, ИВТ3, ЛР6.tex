    % "Лабораторная работа 2"

\documentclass[a4paper,12pt]{article} % тип документа

% report, book

%  Русский язык

\usepackage[T2A]{fontenc}			% кодировка
\usepackage[utf8]{inputenc}			% кодировка исходного текста
\usepackage[english,russian]{babel}	% локализация и переносы


% Математика
\usepackage{amsmath,amsfonts,amssymb,amsthm,mathtools} 


\usepackage{wasysym}

\usepackage{hyperref}

%Заговолок
\author{Иванов Дмитрий, ИВТ3}
\title{Работа с текстом в \LaTeX{}}
\date{\today}


\begin{document} % начало документа
\maketitle
\newpage

\section{Задание 1}
\begin{flushright}
\textbf{1.Для чего предназначена издательская система \LaTeX?}\\
\href{https://ru.wikipedia.org/wiki/LaTeX}{\LaTeX} позволяет автоматизировать многие задачи набора текста и подготовки статей, включая набор текста на нескольких языках, нумерацию разделов и формул, перекрёстные ссылки, размещение иллюстраций и таблиц на странице, ведение библиографии и др.\\
\end{flushright}

\begin{center}
\textbf{2.В каких случаях рационально её использовать?}\\
LaTeX \LARGE{предназначен} \normalsize{для} упрощения процедуры форматирования наиболее распространенных материалов, таких как книги, научные доклады, которые могут содержать множество формул, элементов на разных языках, перекрестных ссылок и цитат, индексов, библиографий. Конечно, все эти вещи можно сделать в LibreOffice, но в LaTexila их выполнить легче, а результат будет выглядеть более качественно.\\
\end{center}

\begin{flushleft}
\textbf{3.Какие преимущества имеет работа в этой системе?}\\
Высокое качество результата, недоступное другим средствам полиграфической подготовки (верстки) текстов. Печать с высоким качеством выполняется как на матричном принтере, так и на фотонаборном автомате.\\
\end{flushleft}

\textbf{4.Какие сложности могут возникнуть при работе в этот системе?}
\begin{enumerate}
\item Требует знания элементарных основ полиграфии. \footnote{Полиграфия — отрасль промышленности, занимающаяся изготовлением печатной, а именно книжно-журнальной, деловой, газетной, этикеточной и упаковочной продукции.}
\item Cложности могут возникнуть при работе с неструктурированными документами. 
\end{enumerate}

\textbf{5.Какие недостатки отмечают пользователи при работе с этой системой?}
\begin{itemize}
\item Работа с исходным текстом и просмотр того, как текст будет выглядеть на печати, — разные операции. 
\item Высокий порог вхождения.
\end{itemize}

\section{Задание 2 и 3}
\begin{enumerate}
\item \textbf{Какие основные преимущества имеет \LaTeX перед обычными текстовыми процессорами?}\\


\begin{itemize}
\item Готовые профессионально выполненные макеты, делающие документы действительно выглядящими "как изданные". 
\item Удобно поддержана верстка математических формул.
\item Пользователю нужно выучить лишь несколько понятных команд, задающих логическую структуру документа. \item Ему практически никогда не нужно возиться собственно с макетом документа.
\item Легко изготавливаются даже сложные структуры, типа примечаний, оглавлений, библиографий и прочее.
\item Существуют свободно распространяемые дополнительные пакеты для многих типографских задач, не поддерживаемых напрямую базовым \LaTeX. Например, наличествуют пакеты для включения POSTSCRIPT графики или для верстки библиографий в точном соответствии с конкретными стандартами.
\item \LaTeX поощряет авторов писать хорошо структурированные документы, так как именно так \LaTeX и работает -- путем спецификации структуры.
\item \TeX, форматирующее сердце \LaTeX, чрезвычайно мобилен и свободно доступен. Поэтому система работает практически на всех существующих платформах.
\end{itemize}

\item \textbf{Какие особенности команд \LaTeX следует учитывать при работе?}\\

Команды  LATEX чувствительны к регистру и принимают одну из следующих двух форм:
\begin{itemize}
\item Они начинаются с символа backslash $\backslash$ и продолжаются именем, состоящим только из букв. Имена команд завершаются пробелом, цифрой или любой другой <<не-буквой>>.
\item Они состоят из $\backslash$ и ровно одного специального символа. LATEX игнорирует пробелы после команд. Если вы хотите получить  пробел после команды, вы должны поместить или \{ \} и пробел, или специальную команду пробела после имени команды. \{ \} не дает LATEX игнорировать все пробелы после имени команды.
\end{itemize}

\item  \textbf{Какие предопределенные стили страницы имеет \LaTeX?}\\

\LaTeX поддерживает три предопределенных комбинации верхнего колонтитула и нижнего колонтитула - так называемые стили страницы:
\begin{itemize}
\item plain - печатает номера страниц внизу страницы в середине нижнего колонтитула. Этот стиль установлен по умолчанию.
\item headings - печатает название текущей главы и номер страницы в верхнем колонтитуле каждой страницы, а нижний колонтитул остается пустым. (Этот стиль использован в данном документе.)
\item empty - делает и верхние, и нижние колонтитулы пустыми.
\end{itemize}

\item  \textbf{Как происходит разбиение строк в \LaTeX?}\\

\LaTeX всегда пытается производить наилучшее из возможных разбиение строк. Если он не может найти способ разбить строки в соответствии со своими стандартами, он позволяет одной строке выступать из абзаца вправо. LATEX затем выводит диагностику ("overfull hbox") во время обработки входного файла. Чаще всего это случается, когда LATEX не может найти место для переноса слова. 

\item  \textbf{Какие команды отвечают за "смягчение" правил переноса строки?}\\

Давая команду sloppy, вы можете сказать, чтобы LATEX несколько ослабил свои стандарты. Тогда он сможет предотвратить такие слишком длинные строки, увеличивая интервалы между словами -- даже, если конечный вывод будет не оптимален. В этом случае пользователь получит предупреждение (<<underfull hbox>>). В большинстве случаев результат выглядит не очень хорошо. 

\item  \textbf{Какие виды тире есть в \LaTeX?}\\

Четыре вида тире. Три из них можно получить различным числом последовательных знаков -. Четвертое на самом деле не тире вовсе, а математический знак минус. Эти тире называются так: - дефис, -- короткое тире, --- длинное тире и $-$ знак минуса.  

\item \textbf{Какое окружение подходит для описаний, простых и нумерованных списков?}\\

Окружение itemize подходит для простых списков, окружение enumerate -- для нумерованных списков, а окружение description -- для описаний.

\item \textbf{Какие окружения лучше использовать для длинных абзацев, охватывающих несколько абзацев, а какое - для стихов, где важны разрывы строк?}\\

Окружение quotation полезно для более длинных цитат, охватывающих несколько абзацев, потому что оно начинает абзацы с красной строки. Окружение verse используют для стихов, где важны разрывы строк. 

\item \textbf{Каким образом можно выровнять числовые столбцы по десятчной точке?}\\

Поскольку встроенный способ выровнять числовые столбцы по десятичной точке отсутствует, мы можем <<обмануть>> TEX и добиться этого при помощи двух столбцов: выровненной вправо целой части и выровненной влево дробной. Команда @{.}  заменяет нормальный пробел между столбцами просто на <<.>>, давая эффект одного столбца, выровненного по десятичной точке. 

\item \textbf{В какой ситуации следует использовать "плавающие объекты"?}\\

Большинство публикаций в наши дни содержат множество иллюстраций и таблиц. Эти элементы нуждаются в специальном обращении с ними, так как они не могут быть разбиты между страницами. Одним из выходов было бы начинать новую страницу каждый раз, когда встречается иллюстрация или таблица, слишком большая, чтобы поместиться на текущей странице. Этот подход привел бы к тому, что страницы оставались бы частично пустыми, что смотрится очень плохо.
\end{enumerate}

\section{Задание 4}

\begin{itemize}
\item \href{https://www.latex-project.org/}{The Latex Project} – ресурс, посвященный \LaTeX, в котором можно найти все новости, документацию и обучающие статьи по системе \LaTeX.
\item \href{https://ru.wikibooks.org/wiki/LaTeX}{Учебник Latex на wikibooks} – русскоязычный подробный учебник по языку TeX, её расширениям (LaTeX, XeLaTeX, и др.), дополнительным инструментам (pdflatex, bibtex и др.) и некоторым пакетам.
\item \href{https://www.coursera.org/learn/latex}{Документы и презентации в LaTeX} - курс на платформе Coursera. На этом курсе можно узнать, как оформить ваши идеи в виде красивого, профессионально сверстанного текста или слайдов презентации.
\item \href{https://www.ibm.com/developerworks/ru/library/latex_tutorial_01/index.html}{Работа в LaTeX. Создание документа на примере подготовки курсовой работы} - цикл статей, посвященный вопросу применения LaTeX для подготовки документа "от начала и до конца".
\item \href{http://www.texdoc.net/}{TexDoc Online} - поисковая система документации по \TeX и \LaTeX.
\end{itemize}

\end{document}