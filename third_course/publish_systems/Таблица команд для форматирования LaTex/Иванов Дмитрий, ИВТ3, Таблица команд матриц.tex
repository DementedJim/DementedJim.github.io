\documentclass[a4paper,12pt]{article} % тип документа
% report, book
%  Русский язык
\usepackage[T2A]{fontenc}			% кодировка
\usepackage[utf8]{inputenc}			% кодировка исходного текста
\usepackage[english,russian]{babel}	% локализация и переносы
% Математика
\usepackage{amsmath,amsfonts,amssymb,amsthm,mathtools} 
\usepackage{wasysym}
\usepackage{array}
\usepackage{hyperref}
%Заговолок
\author{Иванов Дмитрий, ИВТ3}
\title{Таблица  команд матриц в \LaTeX{}}
\date{\today}

\begin{document} % начало документа
\maketitle
\newpage

\flushleft
\begin{tabular}{|l|l|}
  \hline
  \multicolumn{2}{|c|}{Матрицы в LaTex} \\
  \hline
  \backslash \text{команды} \{\text{окружение}\}  & Назначение, примечание \\
    \hline
  begin\{\backslash tabular\}\{|l|l|\} текст \& текст end\{\backslash tabular\} & Таблица\\
  \hline
  begin\{\backslash array\}\{clr\} 1 \& 2 end\{\backslash array\} & Матрица \\
  \hline
  \begin{matrix}\backslash begin\{pmatrix\}
\backslash alpha \& & \backslash beta \backslash\backslash\\
\backslash gamma \& & \backslash delta
\backslash end\{pmatrix\} \end{matrix}  & Круглые скобки:
\begin{pmatrix}
  \alpha& \beta\\
  \gamma& \delta
\end{pmatrix} \\
  \hline
    \begin{matrix}\backslash begin\{Bmatrix\}
\backslash alpha \& & \backslash beta \backslash\backslash\\
\backslash gamma \& & \backslash delta
\backslash end\{Bmatrix\} \end{matrix}  & Фигурные скобки:
\begin{Bmatrix}
\alpha& \beta\\
\gamma& \delta
\end{Bmatrix}\\
  \hline
  \begin{matrix}\backslash begin\{bmatrix\}
\backslash alpha \& & \backslash beta \backslash\backslash\\
\backslash gamma \& & \backslash delta
\backslash end\{bmatrix\} \end{matrix}  & Квадратными скобками:
\begin{bmatrix}
\alpha& \beta\\
\gamma& \delta
\end{bmatrix}\\
  \hline
  \begin{matrix}\backslash begin\{matrix\}
\backslash alpha \& & \backslash beta \backslash\backslash\\
\backslash gamma \& & \backslash delta
\backslash end\{matrix\} \end{matrix}  &  Матрица без скобок:
\begin{matrix}
\alpha& \beta\\
\gamma& \delta
\end{matrix} \\
 \hline
  \begin{matrix}\backslash begin\{vmatrix\}
\backslash alpha \& & \backslash beta \backslash\backslash\\
\backslash gamma \& & \backslash delta
\backslash end\{vmatrix\} \end{matrix}  & Одинарный модуль:
\begin{vmatrix}
  \alpha& \beta\\
  \gamma& \delta
\end{vmatrix}\\
  \hline
   \begin{matrix}\backslash begin\{Vmatrix\}
\backslash alpha \& & \backslash beta \backslash\backslash\\
\backslash gamma \& & \backslash delta
\backslash end\{Vmatrix\} \end{matrix}  & Двойной модуль (норма):
\begin{Vmatrix}
  \alpha& \beta^{*}\\
  \gamma^{*}& \delta
\end{Vmatrix}\\
  \hline
\end{tabular}

\end{document} % конец документа