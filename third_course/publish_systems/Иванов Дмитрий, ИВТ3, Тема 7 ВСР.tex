    % "Лабораторная работа 2"

\documentclass[a4paper,12pt]{article} % тип документа

% report, book

%  Русский язык

\usepackage[T2A]{fontenc}			% кодировка
\usepackage[utf8]{inputenc}			% кодировка исходного текста
\usepackage[english,russian]{babel}	% локализация и переносы


% Математика
\usepackage{amsmath,amsfonts,amssymb,amsthm,mathtools} 


\usepackage{wasysym}

\usepackage{hyperref}

%Заговолок
\author{Иванов Дмитрий, ИВТ3}
\title{Особенности создания текста с формулами \LaTeX{}}
\date{\today}


\begin{document} % начало документа
\maketitle
\newpage

\section{Глава 1. Предел последовательности}
\subsection{Определение предела последовательности}
Напомним определение последовательности элементов произвольного множества A, данное в главе 1. Если каждому натуральному числу n поставлен в соответствие некоторый элемент $x_n$ из множества A, то говорят, что элементы $$x_1, x_2, x_3 \ldots$$ образуют последовательность. 

В этой главе рассматриваются в основном числовые последовательности. Для краткости будем называть их просто последовательностями.
Определение. Число $a$ называют \textit{пределом последовательности}, если для каждого положительного числа $\epsilon$ существует число $N = N(\epsilon)$ такое, что при всех $n > N$ выполняется неравенство $$|x_n - a| < \epsilon$$  В этом случае пишут $$a = \lim_{n\to \infty} x_n = \lim_{n} x_n = \lim x_n$$ или $$x_n\to a,n\to \infty.$$

Для a = 0 это определение было дано в § 1.7, когда говорилось
о последовательностях, сходящихся к нулю.

Определение. Интервал $(a - \epsilon, a + \epsilon)$, где $ε > 0$, называют
$\epsilon$-окрестностью точки a.

С помощью $\epsilon$-окрестностей, определение предела можно сформулировать так: число $a$ называется пределом последовательности, если для каждого положительного числа $\epsilon$ все члены последовательности, начиная с некоторого, принадлежат $\epsilon$-окрестности точки $a$.

Определение. Если последовательность имеет предел, её называют сходящейся. Последовательности, не имеющие предела, называют расходящимися.

Если a является пределом последовательности $\{x_n\}$, то говорят, что последовательность сходится к $a$. О членах последовательности (числах $x_n$) говорят, что они сходятся или стремятся к $a$.

\end{document}