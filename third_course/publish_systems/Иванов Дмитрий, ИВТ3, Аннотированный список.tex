    % "Лабораторная работа 2"

\documentclass[a4paper,12pt]{article} % тип документа

% report, book

%  Русский язык

\usepackage[T2A]{fontenc}			% кодировка
\usepackage[utf8]{inputenc}			% кодировка исходного текста
\usepackage[english,russian]{babel}	% локализация и переносы


% Математика
\usepackage{amsmath,amsfonts,amssymb,amsthm,mathtools} 


\usepackage{wasysym}

\usepackage{hyperref}

%Заговолок
\author{Иванов Дмитрий, ИВТ3}
\title{Работа с текстом в \LaTeX{}}
\date{\today}


\begin{document} % начало документа
\maketitle
\newpage


\section{Аннотированный список ресурсов, содержащих рекомендации по использованию LaTeX}

\begin{itemize}
\item \href{https://www.latex-project.org/}{The Latex Project} – ресурс, посвященный \LaTeX, в котором можно найти все новости, документацию и обучающие статьи по системе \LaTeX.
\item \href{https://ru.wikibooks.org/wiki/LaTeX}{Учебник Latex на wikibooks} – русскоязычный подробный учебник по языку TeX, её расширениям (LaTeX, XeLaTeX, и др.), дополнительным инструментам (pdflatex, bibtex и др.) и некоторым пакетам.
\item \href{https://www.coursera.org/learn/latex}{Документы и презентации в LaTeX} - курс на платформе Coursera. На этом курсе можно узнать, как оформить ваши идеи в виде красивого, профессионально сверстанного текста или слайдов презентации.
\item \href{https://www.ibm.com/developerworks/ru/library/latex_tutorial_01/index.html}{Работа в LaTeX. Создание документа на примере подготовки курсовой работы} - цикл статей, посвященный вопросу применения LaTeX для подготовки документа "от начала и до конца".
\item \href{http://www.texdoc.net/}{TexDoc Online} - поисковая система документации по \TeX и \LaTeX.
\end{itemize}

\section{Аннотированный список онлайн сервисов LaTex}

\begin{itemize}
\item \href{https://overleaf.com/}{Overleaf} – простой онлайн редактор LaTeX, с поддержкой совместной работы и большинства возможностей LaTeX, включая вставку изображений, библиографии, формулы и многих других.
\item \href{https://ru.sharelatex.com/}{ShareLaTeX} - Overleaf и ShareLaTeX объединили усилия, чтобы создать v2, который сочетает в себе лучшее из обоих сервисов. Исходя из ShareLaTeX, Overleaf v2 будет очень знакомым, поскольку он построен на основе редактора ShareLaTeX.
\item \href{https://papeeria.com/}{Papeeria} - Онлайн сервис по созданию файлов в LaTeX и разметки markdown с мобильной версией и богатой базой готовых шаблонов
\item \href{https://www.codecogs.com}{Codecogs} - русскоязычный онлайн редактор формул LaTeX
\item \href{https://latexbase.com}{LaTeXBase} - простой онлайн редактор LaTeX, не требующий регистрации 
\item \href{https://www.tutorialspoint.com/}{CodingGround} - онлайн редактор LaTeX с продвинутой панелью инструментов
\end{itemize}
\end{document}