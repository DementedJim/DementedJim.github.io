\documentclass[a4paper,12pt]{article} % тип документа

% report, book

%  Русский язык

\usepackage[T2A]{fontenc}			% кодировка
\usepackage[utf8]{inputenc}			% кодировка исходного текста
\usepackage[english,russian]{babel}	% локализация и переносы


% Математика
\usepackage{amsmath,amsfonts,amssymb,amsthm,mathtools} 


\usepackage{wasysym}

\usepackage{hyperref}

%Заговолок
\author{Иванов Дмитрий, ИВТ3}
\title{Создание матриц средствами \LaTeX}
\date{\today}


\begin{document} % начало документа
\maketitle
\newpage
\section{Задание 1}
\textbf{Пример 1. Умножение матрицы на число} \\
\small{\textbf{Дано:}}

\begin{displaymath}
\mathbf{A} =
\left( \begin{array}{ccc}
1 & 2 & 3 \\
4 & 5 & 6 \\
\end{array} \right)
\end{displaymath}
Матрица A\\
Число $k=2$.\\
\small{\textbf{Найти:}}\\
Произведение матрицы на число: $A \times k = B$\\
B - ?\\
\textbf{Решение:}
\begin{flushleft}
Для того чтобы умножить матрицу A на чисо $k$ нужно каждый элемент\\
матрицы A умножить на это число.\\
Таким образом, произведение матрицы A на число k есть новая матрица:
\end{flushleft}


\begin{displaymath}    
B = 2 \times A = 2 \times  \left( \begin{array}{ccc}
1 & 2 & 3 \\
4 & 5 & 6 \\
\end{array} \right) = \left( \begin{array}{ccc}
2 & 4 & 6 \\
8 & 10 & 12 \\
\end{array} \right) \\ 
$$
\end{displaymath}

\textbf{Ответ:}
$\mathbf{B} =  \left( \begin{array}{ccc}
2 & 4 & 6 \\
8 & 10 & 12 \\\end{array} \right)$

\newpage

\section{Задание 2}
\textbf{Пример 2. Умножение Матриц} \\
\small{\textbf{Дано:}}


Матрциа $\mathbf{A} =
\left( \begin{array}{ccc}
2 & 3 & 1 \\
-1 & 0 & 1 \\
\end{array} \right)\\$

Матрица $\mathbf{B} =
\left( \begin{array}{cc}
2 & 1  \\
-1 & 1 \\
3 & -2 \\
\end{array} \right)\\$
\small{\textbf{Найти:}}\\
Произведение матрицы на число: A \times B = C \\
C - ?\\
\small{\textbf{Решение:}}
\begin{flushleft}
\footnotesize Каждый элемент матрицы $C = A \times B$ , расположенный в $i$-й строке и $j$-м столбце,
равен сумме произведений элементов $i$-й строки матрицы A на соответвующие \\
элементы $j$-го стобца матрицы B. Строки матрицы A умножаем на столбцы \\
матрицы B и получаем: 
\end{flushleft}
\begin{displaymath}    
\\
C = A \times B =\left( \begin{array}{ccc}
2 & 3 & 1 \\
-1 & 0 & 1 \\
\end{array} \right) \times \left( \begin{array}{cc}
2 & 1  \\
-1 & 1 \\
3 & -2 \\
\end{array} \right) = \\
= \left( \begin{array}{cc}
2 \times 2 + 3 \times (-1) + 1 \times 3 & 2 \times 1 + 3 \times 1 + 1 \times (-1)  \\
-1 \times 2 + 0 \times (-1) + 1 \times 3 & -1 \times 1 + 0 \times 1 + 1 \times (-2)\\
\end{array} \right)\\ 
С = A \times B =  \left( \begin{array}{cc}
4 & 3 \\
1 & -3\\
\end{array} \right)
\end{displaymath}

\textbf{Ответ:} 
$\mathbf{C} =  \left( \begin{array}{cc}
4 & 3 \\
1 & -3\\
\end{array} \right)
$
   
\newpage

\section{Задание 3}
\textbf{Пример 3. Транспанирование матриц} \\
\small{\textbf{Дано:}}

Матрциа $\mathbf{A} =
\left( \begin{array}{ccc}
7 & 8 & 9 \\
1 & 2 & 3 \\
\end{array} \right)$\\

\small{\textbf{Найти:}}\\
Найти матрицу транспонированную данной.\\
$A^{T} -?$ \\
\small{\textbf{Решение:}}\\
Транспонирование матрицы A заключается в замене строк матрицы ее
столбцами с сохранением номеров. Полученная матрица обозначается\\
через $A^{T}$

\begin{displaymath}
\\
\mathbf{A} =
\left( \begin{array}{ccc}
7 & 8 & 9 \\
1 & 2 & 3 \\
\end{array} \right) \Rightarrow A^{T} = \left( \begin{array}{cc}
7 & 1  \\
8 & 2 \\
9 & 3\\
\end{array} \right)
\end{displaymath}

\small{\textbf{Ответ:}}  $A^{T} = \left( \begin{array}{cc}
7 & 1  \\
8 & 2 \\
9 & 3\\
\end{array} \right)$

\newpage

\section{Задание 4}
\textbf{Пример 4. Обратная матрица} \\
\small{\textbf{Дано:}} \\
Матрица $\mathbf{A} =
\left( \begin{array}{cc}
2 & -1  \\
3 & 1 \\
\end{array} \right)$ \\

\small{\textbf{Найти:}} \\
Найти обратную матрицу для матрицы A.\\
A^{-1} - ? \\

\small{\textbf{Решение:}} \\

Находим $\det$ A и проверяем $\det A \neq 0$:
\begin{displaymath}
\\
\det A = \begin{bmatrix}
2 & -1\\
3 & 1 \\
\end{bmatrix} = 2 \times 1 - 3 \times (-1) = 5
\end{displaymath}\\

\footnotesize{$\det A = 5 \neq 0.$}\\

Составляем вспомогательную матрицу $A^{V}$ из алгебраичиских\\
дополнений $A_{i}{j}$:
\begin{displaymath}
    A^{V} = \left(\begin{array}{cc}
    1 & -3  \\
    1 & 2 \\
  \end{array} \right)
\end{displaymath}\\
Транспонируем матрицу A$^{V}$:
\begin{displaymath}
\\
    \left(A^{V} \right)^{T}  = \left(\begin{array}{cc}
    1 & 1  \\
    -3 & 2 \\
  \end{array} \right)
\end{displaymath} \\

Каждый элемент, полученной матрицы, делим на \det A: 
\begin{displaymath}
\\
    A^{-1}  = \frac{1}{\det A}\left(A^{V}\right)^{T} = \frac{1}{5} \times
    \left(\begin{array}{cc}
    1 & 1  \\
    -3 & 2 \\
  \end{array} \right) = 
   \left(\begin{array}{cc}
    \frac{1}{5} & \frac{1}{5}  \\
    \frac{3}{5} & \frac{2}{5} \\
  \end{array} \right) 
\end{displaymath} \\

\textbf{Ответ:} $ \mathbf{A}^{-1} = \left(\begin{array}{cc}
    \frac{1}{5} & \frac{1}{5}  \\
    \frac{3}{5} & \frac{2}{5} \\
  \end{array} \right) $
\end{document}